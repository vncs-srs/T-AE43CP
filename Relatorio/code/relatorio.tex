\documentclass[relatorio]{IEEEtran}
\IEEEoverridecommandlockouts
% The preceding line is only needed to identify funding in the first footnote. If that is unneeded, please comment it out.
\usepackage{cite}
\usepackage{amsmath,amssymb,amsfonts}
\usepackage{algorithmic}
\usepackage{graphicx}
\usepackage{textcomp}
\usepackage{xcolor}
\usepackage{listings}

\lstdefinestyle{myCStyle}{
    language=C,
    commentstyle=\color{green},
    keywordstyle=\color{blue},
    numberstyle=\tiny\color{gray},
    stringstyle=\color{red},
    basicstyle=\ttfamily\small,
    breakatwhitespace=false,         
    breaklines=true,                 
    captionpos=b,                    
    keepspaces=true,                 
    numbers=left,                    
    numbersep=5pt,                  
    showspaces=false,                
    showstringspaces=false,
    showtabs=false,                  
    tabsize=2
}

\lstset{style=myCStyle}

\def\BibTeX{{\rm B\kern-.05em{\sc i\kern-.025em b}\kern-.08em
    T\kern-.1667em\lower.7ex\hbox{E}\kern-.125emX}}
\begin{document}

\title{Movimento do Cavalo\\
{\footnotesize \textsuperscript{}Trabalho de Algoritimos e Estrutura de Dados 2}
}

\author{\IEEEauthorblockN{1\textsuperscript{st} Caetano Chinarelli Souza}
\IEEEauthorblockA{\textit{Regitro Acadêmico 2344955 } \\
\textit{Universidade Tecnológica Federal do Paraná }\\
Pato Branco, Paraná \\
}
\and
\IEEEauthorblockN{2\textsuperscript{nd} Guilherme Iago Marcante Della Libera}
\IEEEauthorblockA{\textit{Regitro Acadêmico 2199572} \\
\textit{Universidade Tecnológica Federal do Paraná }\\
Pato Branco, Paraná \\
}
\and
\IEEEauthorblockN{3\textsuperscript{rd} Kelvyn Augusto Waltrick Nonato}
\IEEEauthorblockA{\textit{Regitro Acadêmico 2345048} \\
\textit{Universidade Tecnológica Federal do Paraná }\\
Pato Branco, Paraná \\
}
\and
\IEEEauthorblockN{4\textsuperscript{th} Luiz Eduardo Rufatto }
\IEEEauthorblockA{\textit{Regitro Acadêmico  2079933} \\
\textit{Universidade Tecnológica Federal do Paraná }\\
Pato Branco, Paraná \\
}
\and
\IEEEauthorblockN{5\textsuperscript{th} Vinicius Soares do Rosario}
\IEEEauthorblockA{\textit{Regitro Acadêmico  2247305} \\
\textit{Universidade Tecnológica Federal do Paraná }\\
Pato Branco, Paraná \\
}
}

\maketitle

\section{Introdução}
Texto aqui

\section{Problemática}

\subsection{Descrição do problema}
Deslocamento do cavalo: dado um tabuleiro de xadrez NxN, uma posição inicial (x0, y0) e uma posição final (xf, yf). A partir da posição
inicial, encontrar, caso existir, um passeio com uma quantidade mínima de passos para chegar até a posição final. Em seguida, devem
ser impresso a quantidade de passos e uma matriz, onde cada elemento deve indicar o número de passos para chegar em tal posição.
Por exemplo, na posição (x0, y0) deve ser impresso 0 e, no próximo passo, 1, e assim por diante. Obs.: cada passo deve seguir a regra
de xadrez para o movimento do cavalo.
\subsection{Motivação para a escolha do problema}
Texto aqui

\subsection{Estrategias para a solução}
Texto aqui

\section{Descrição das Soluções do Problema}
Texto aqui
\section{Análise de Complexidade Tempo e de Espaço}
Texto aqui
\begin{lstlisting}[language=C]
    #include <stdio.h>
    int main() {
        printf("Ola, Mundo!\n");
        return 0;
    }
\end{lstlisting}

\begin{table}[htbp]
\caption{Table Type Styles}
\begin{center}
\begin{tabular}{|c|c|c|c|}
\hline
\textbf{Table}&\multicolumn{3}{|c|}{\textbf{Table Column Head}} \\
\cline{2-4} 
\textbf{Head} & \textbf{\textit{Table column subhead}}& \textbf{\textit{Subhead}}& \textbf{\textit{Subhead}} \\
\hline
copy& More table copy$^{\mathrm{a}}$& &  \\
\hline
\multicolumn{4}{l}{$^{\mathrm{a}}$Sample of a Table footnote.}
\end{tabular}
\label{tab1}
\end{center}
\end{table}

\begin{figure}[htbp]
\centerline{\includegraphics{fig1.png}}
\caption{Example of a figure caption.}
\label{fig}
\end{figure}


\section{Conclusão}
Texto aqui

\section{Declaração de Autoria}
Relatorio dos membros aqui

\section*{Referências}

\begin{thebibliography}{00}
\bibitem{b1} G. Eason, B. Noble, and I. N. Sneddon, ``On certain integrals of Lipschitz-Hankel type involving products of Bessel functions,'' Phil. Trans. Roy. Soc. London, vol. A247, pp. 529--551, April 1955.
\bibitem{b2} J. Clerk Maxwell, A Treatise on Electricity and Magnetism, 3rd ed., vol. 2. Oxford: Clarendon, 1892, pp.68--73.
\bibitem{b3} I. S. Jacobs and C. P. Bean, ``Fine particles, thin films and exchange anisotropy,'' in Magnetism, vol. III, G. T. Rado and H. Suhl, Eds. New York: Academic, 1963, pp. 271--350.
\bibitem{b4} K. Elissa, ``Title of paper if known,'' unpublished.
\bibitem{b5} R. Nicole, ``Title of paper with only first word capitalized,'' J. Name Stand. Abbrev., in press.
\bibitem{b6} Y. Yorozu, M. Hirano, K. Oka, and Y. Tagawa, ``Electron spectroscopy studies on magneto-optical media and plastic substrate interface,'' IEEE Transl. J. Magn. Japan, vol. 2, pp. 740--741, August 1987 [Digests 9th Annual Conf. Magnetics Japan, p. 301, 1982].
\bibitem{b7} M. Young, The Technical Writer's Handbook. Mill Valley, CA: University Science, 1989.
\end{thebibliography}
\vspace{12pt}
\color{red}
texto aqui
\end{document}
