\documentclass[relatorio]{IEEEtran}
\IEEEoverridecommandlockouts
% The preceding line is only needed to identify funding in the first footnote. If that is unneeded, please comment it out.
\usepackage{cite}
\usepackage{amsmath,amssymb,amsfonts}
\usepackage{algorithmic}
\usepackage{graphicx}
\usepackage{textcomp}
\usepackage{xcolor}
\usepackage{listings}

\lstdefinestyle{myCStyle}{
    language=C,
    commentstyle=\color{green},
    keywordstyle=\color{blue},
    numberstyle=\tiny\color{gray},
    stringstyle=\color{red},
    basicstyle=\ttfamily\small,
    breakatwhitespace=false,         
    breaklines=true,                 
    captionpos=b,                    
    keepspaces=true,                 
    numbers=left,                    
    numbersep=5pt,                  
    showspaces=false,                
    showstringspaces=false,
    showtabs=false,                  
    tabsize=2
}

\lstset{style=myCStyle}

\def\BibTeX{{\rm B\kern-.05em{\sc i\kern-.025em b}\kern-.08em
    T\kern-.1667em\lower.7ex\hbox{E}\kern-.125emX}}
\begin{document}

\title{Movimento do Cavalo\\
{\footnotesize \textsuperscript{}Trabalho de Algoritimos e Estrutura de Dados 2}
}

\author{\IEEEauthorblockN{1\textsuperscript{st} Caetano Chinarelli Souza}
\IEEEauthorblockA{\textit{Regitro Acadêmico 2344955 } \\
\textit{Universidade Tecnológica Federal do Paraná }\\
Pato Branco, Paraná \\
}
\and
\IEEEauthorblockN{2\textsuperscript{nd} Guilherme Iago Marcante Della Libera}
\IEEEauthorblockA{\textit{Regitro Acadêmico 2199572} \\
\textit{Universidade Tecnológica Federal do Paraná }\\
Pato Branco, Paraná \\
}
\and
\IEEEauthorblockN{3\textsuperscript{rd} Kelvyn Augusto Waltrick Nonato}
\IEEEauthorblockA{\textit{Regitro Acadêmico 2345048} \\
\textit{Universidade Tecnológica Federal do Paraná }\\
Pato Branco, Paraná \\
}
\and
\IEEEauthorblockN{4\textsuperscript{th} Luiz Eduardo Rufatto }
\IEEEauthorblockA{\textit{Regitro Acadêmico  2079933} \\
\textit{Universidade Tecnológica Federal do Paraná }\\
Pato Branco, Paraná \\
}
\and
\IEEEauthorblockN{5\textsuperscript{th} Vinicius Soares do Rosario}
\IEEEauthorblockA{\textit{Regitro Acadêmico  2247305} \\
\textit{Universidade Tecnológica Federal do Paraná }\\
Pato Branco, Paraná \\
}
}

\maketitle

\section{Introdução}
Com o tempo, a programação foi se desenvolvendo cada vez mais rápido ao redor do mundo, e com isso foi se criando das mais diversas técnicas para a resolução de muitos problemas existente, e com isso conseguindo cada vez mais a otimização de códigos para seu melhor funcionamento e execução.

No problema que foi escolhido para resolvermos, o deslocamento de um cavalo em um tabuleiro de xadrez, em que é necessário achar a melhor combinação de passos possíveis para chegar até a posição final definida pelo jogador. Com isso, se encaixa a otimização de resultados, fazendo com que o processamento seja cada vez menor, buscando o melhor caso entre todos os possíveis.

Ao fazer a escolha do melhor modelo para resolver o problema selecionado, foi analisado qual seria a ideia para melhor aproveitamento de memória e processamento, onde o backtraking acabou se destacando por ser um modelo que descarta soluções mediana ou péssimas sem mesmo terminar sua checagem, e escolhendo sempre o melhor caso para a resolução do problema.

Após o problema resolvido e a concretização da melhor ação a ser tomada, a possibilidade da resolução de outros tipos de problema desse meio utilizando essa mesma técnica poderá ser feita com muito mais rapidez e economizando memória e processamento da máquina que está sendo utilizada.

\section{Problemática}

\subsection{Descrição do problema}
O problema tratado foi o do deslocamento do cavalo no tabuleiro de xadrez. Esta proposta, segue a seguinte descrição (conforme consta nas especificações do trabalho): dado um tabuleiro de ordem N (NxN) e uma posição inicial (X0, Y0), o algoritmo deve encontrar uma solução ótima (caso exista) com a menor quantidade de movimentos para chegar a posição final. O padrão de movimento do cavalo em um jogo de xadrez normal deve ser seguido, isto é, “em L”. Além disso, deve ser impresso a quantidade de passos e uma matriz onde cada elemento indica o número de passos para chegar a posição em questão.

A escolha deste problema resumiu-se a um acordo comum entre todos os integrantes do grupo através de uma votação, levando em consideração o consenso entre os membros que este foi de fácil compreensão, o que permitiu um planejamento mais claro e eficiente quanto a como abordá-lo. 

\subsection{Motivação para a escolha do problema}
Texto aqui

\subsection{Estrategias para a solução}
Do mesmo modo que são tratados diversos problemas envolvendo algoritmos, é possível encontrar múltiplas soluções através de diferentes estratégias aplicadas. O critério de desempate, neste caso e em geral, reduz-se a escolha do algoritmo com melhor eficiência em comparação aos demais, ou seja, que possuí o menor custo computacional. Ainda para a escolha, consideram-se a facilidade de manipular a estratégia para a aplicação desejada e a legibilidade do código. A partir disso, foi decidido a escolha de uma estratégia comprovadamente eficiente para a resolução do cenário em questão: backtracking. 
Backtracking é um tipo de algoritmo que deriva da busca por força bruta, sendo um refinamento desta. A estratégia em questão é utilizada para encontrar todas ou algumas soluções para problemas computacionais, especialmente problemas de satisfação de restrições. Consiste em construir candidatos para as soluções, “abandonando” um candidato (origina o termo backtrack, traduzindo, “retroceder”) ao determinar que não pode ser completado para uma solução válida. Isto implica que o algoritmo só é válido para problemas que admitem “soluções parciais” como candidatos para soluções, então possibilitando um teste de validez relativamente simples para que a solução possa ser completada.
Conceitualmente, backtracking é frequentemente representado por uma estrutura de árvore, seguindo o padrão de busca em profundidade. Os candidatos a solução são os nós da árvore e cada nó de solução parcial é nó pai dos nós candidatos a solução, portanto também são nós intermediários. Se um nó candidato parcial é determinado como inválido, toda a subárvore desse nó também é desconsiderada, então a busca retrocede para o nó anterior e repete o processo de verificação para os demais nós da estrutura. A árvore é percorrida recursivamente, garantindo que toda a árvore válida é percorrida, mas não toda a árvore potencial (que contém todos os possíveis candidatos mas nem todos válidos).

\section{Descrição das Soluções do Problema}
Texto aqui
\section{Análise de Complexidade Tempo e de Espaço}
Texto aqui
\begin{lstlisting}[language=C]
    #include <stdio.h>
    int main() {
        printf("Ola, Mundo!\n");
        return 0;
    }
\end{lstlisting}

\begin{table}[htbp]
\caption{Table Type Styles}
\begin{center}
\begin{tabular}{|c|c|c|c|}
\hline
\textbf{Table}&\multicolumn{3}{|c|}{\textbf{Table Column Head}} \\
\cline{2-4} 
\textbf{Head} & \textbf{\textit{Table column subhead}}& \textbf{\textit{Subhead}}& \textbf{\textit{Subhead}} \\
\hline
copy& More table copy$^{\mathrm{a}}$& &  \\
\hline
\multicolumn{4}{l}{$^{\mathrm{a}}$Sample of a Table footnote.}
\end{tabular}
\label{tab1}
\end{center}
\end{table}

\begin{figure}[htbp]
\centerline{\includegraphics{fig1.png}}
\caption{Example of a figure caption.}
\label{fig}
\end{figure}


\section{Conclusão}
Texto aqui

\section{Declaração de Autoria}
Relatorio dos membros aqui

\section*{Referências}

\begin{thebibliography}{00}
\bibitem{b1} G. Eason, B. Noble, and I. N. Sneddon, ``On certain integrals of Lipschitz-Hankel type involving products of Bessel functions,'' Phil. Trans. Roy. Soc. London, vol. A247, pp. 529--551, April 1955.
\bibitem{b2} J. Clerk Maxwell, A Treatise on Electricity and Magnetism, 3rd ed., vol. 2. Oxford: Clarendon, 1892, pp.68--73.
\bibitem{b3} I. S. Jacobs and C. P. Bean, ``Fine particles, thin films and exchange anisotropy,'' in Magnetism, vol. III, G. T. Rado and H. Suhl, Eds. New York: Academic, 1963, pp. 271--350.
\bibitem{b4} K. Elissa, ``Title of paper if known,'' unpublished.
\bibitem{b5} R. Nicole, ``Title of paper with only first word capitalized,'' J. Name Stand. Abbrev., in press.
\bibitem{b6} Y. Yorozu, M. Hirano, K. Oka, and Y. Tagawa, ``Electron spectroscopy studies on magneto-optical media and plastic substrate interface,'' IEEE Transl. J. Magn. Japan, vol. 2, pp. 740--741, August 1987 [Digests 9th Annual Conf. Magnetics Japan, p. 301, 1982].
\bibitem{b7} M. Young, The Technical Writer's Handbook. Mill Valley, CA: University Science, 1989.
\end{thebibliography}
\vspace{12pt}
\color{red}
texto aqui
\end{document}
